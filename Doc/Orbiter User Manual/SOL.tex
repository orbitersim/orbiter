\documentclass[Orbiter User Manual.tex]{subfiles}
\begin{document}

\section{The solar system neighbourhood}
Orbiter's playground is our solar system. The basic installation includes its major bodies (the Sun and planets) and a subset of minor bodies (moons and asteroids), providing a variety of mission targets. Planning and executing a trip that is efficient both in duration and fuel consumption offers different challenge levels. Additional objects may be downloaded as add-ons.

\begin{figure}[H]
	\centering
	\includegraphics[width=0.99\hsize]{sol.png}
\end{figure}


\subsection{Solar system bodies}

\begin{table}[H]
	\begin{tabularx}{\textwidth}{ |lX| }
	\hline\rule{0pt}{2ex}
	\textbf{Sun} &\\
	\hline\rule{0pt}{2ex}
	\adjustbox{valign=t}{
		\begin{tabular}{ c }
		\includegraphics[width=0.6\textwidth, margin=-10pt 1ex -10pt 1ex, valign=m]{solsys_sun.jpg}\\
		\end{tabular}
		}
	& \vfill
	The Sun is a type G2 main sequence star at the centre of our solar system, formed approximately 4.6 billion years ago from a gravitationally collapsing interstellar gas cloud. It comprises nearly the entire mass of the solar system and emits a large amount of electromagnetic energy generated by nuclear fusion in its core. In another 5 billion years the hydrogen fuel will be exhausted and the Sun will expand into a red giant, losing a significant amount of its mass, before the remaining core, a white dwarf, will slowly cool down over trillions of years.\\
	\hline
	\end{tabularx}
\end{table}


\begin{table}[H]
	\begin{tabularx}{\textwidth}{ |lX| }
	\hline\rule{0pt}{2ex}
	\textbf{Mercury} &\\
	\hline\rule{0pt}{2ex}
	\adjustbox{valign=t}{
		\begin{tabular}{ c }
		\includegraphics[width=0.6\textwidth, margin=-10pt 1ex -10pt 1ex, valign=m]{solsys_mercury.jpg}\\
		\end{tabular}
		}
	& \vfill
	Mercury is one of the four terrestrial (rocky) inner planets of the solar system. Its orbit is the closest to the Sun and the most eccentric of all planets.\newline
		Mercury has no significant atmosphere, and its slow rotation (it is tidally locked with the Sun in a 3:2 spin-orbit resonance) results in a large variation between day and night surface temperatures. The surface is heavily cratered and similar in appearance to Earth's Moon.\newline
		\newline
		\textit{Mercury's visual representation in Orbiter is based on images from the Messenger mission.}\\
	\hline
	\end{tabularx}
\end{table}


\begin{table}[H]
	\begin{tabularx}{\textwidth}{ |lX| }
	\hline\rule{0pt}{2ex}
	\textbf{Venus} &\\
	\hline\rule{0pt}{2ex}
	\adjustbox{valign=t}{
		\begin{tabular}{ c }
		\includegraphics[width=0.6\textwidth, margin=-10pt 1ex -10pt 1ex, valign=m]{solsys_venus.jpg}\\
		\end{tabular}
		}
	& \vfill
	Venus is the second planet from the Sun, and similar in mass and size to Earth. It has a very dense atmosphere of mainly carbon dioxide that completely obscures the surface at visible wavelengths and results in a high albedo. The extreme surface pressure and temperatures pose severe challenges for surface-bound missions.\newline
		\newline
		\textit{In Orbiter, the surface representation of Venus is based on the synthetic aperture radar data from the Magellan mission, while the cloud layer is derived from a map by Björn Jónsson.}\\
	\hline
	\end{tabularx}
\end{table}


\begin{table}[H]
	\begin{tabularx}{\textwidth}{ |lX| }
	\hline\rule{0pt}{2ex}
	\textbf{Earth} &\\
	\hline\rule{0pt}{2ex}
	\adjustbox{valign=t}{
		\begin{tabular}{ c }
		\includegraphics[width=0.6\textwidth, margin=-10pt 1ex -10pt 1ex, valign=m]{solsys_earth.jpg}\\
		\includegraphics[width=0.6\textwidth, margin=-10pt 1ex -10pt 1ex, valign=m]{solsys_moon.jpg}\\
		\end{tabular}
		}
	& \vfill
	The home planet. Earth is the only place we know to date for life to have evolved. Its distance from the Sun, atmospheric conditions and internal heating allow for liquid water to exist in large quantities, forming its oceans. Its atmospheric layer, as well as its magnetic field, protect the surface from harmful radiation. The bio­sphere (the sections of surface, lower atmosphere and oceans containing life) is a self-regulating system whose balance may be threatened by human activity (pollution, release of greenhouse gases).\newline
	The Moon is Earth's natural satellite. It has no significant atmosphere, and its surface is characterised by impact craters. The Earth-Moon system is unusual in the solar system for the relatively large size of the Moon compared to the planet it orbits, and may have an influence on stabilising Earth's spin axis.\newline
	\newline
	\textit{In Orbiter, Earth's surface textures have been custom processed from Landsat-7 ETM orthorectified scenes, augmented by aerial imagery for selected areas. Elevations are from NASA SRTM 90m Digital elevation data. Moon surface textures were generated from LRO LRO-WAC Global Mosaic 100m, elevations from LOLA-GDR/Cylindrical}\\
	\hline
	\end{tabularx}
\end{table}


\begin{table}[H]
	\begin{tabularx}{\textwidth}{ |lX| }
	\hline\rule{0pt}{2ex}
	\textbf{Mars} &\\
	\hline\rule{0pt}{2ex}
	\adjustbox{valign=t}{
		\begin{tabular}{ c }
		\includegraphics[width=0.6\textwidth, margin=-10pt 1ex -10pt 1ex, valign=m]{solsys_mars.jpg}\\
			\adjustbox{valign=t}{
			\begin{tabular}{ ll }
			\includegraphics[width=0.285\textwidth, margin=-20pt 1ex 0pt 1ex, valign=m]{solsys_phobos.jpg} &
			\includegraphics[width=0.285\textwidth, margin=0pt 1ex -20pt 1ex, valign=m]{solsys_deimos.jpg}\\
			\end{tabular}
			}
		\end{tabular}
		}
	& \vfill
	Mars is smaller than Earth (about 1/10 of its mass) with a very thin atmosphere predominantly consisting of carbon dioxide, which nevertheless can produce extensive dust storms that sometimes cover the entire surface. Geological features include Valles Marineris, a 3000 km long network of canyons, and Olympus Mons, the largest shield volcano in the solar system. The polar caps are mainly composed of solid carbon dioxide with small quantities of water ice. Mars has two small irregularly shaped moons, Phobos and Deimos. There have been numerous unmanned missions to Mars, including several rovers to explore the surface.\newline
	\newline
	\textit{The Mars surface textures and elevation data in Orbiter are composed from ESA Mars Express High Resolution Stereo Camera (HRSC) imagery, with missing areas filled from NASA MOC (surface) and MOLA (elevation) data.}\\
	\hline
	\end{tabularx}
\end{table}


\begin{table}[H]
	\begin{tabularx}{\textwidth}{ |lX| }
	\hline\rule{0pt}{2ex}
	\textbf{Vesta} &\\
	\hline\rule{0pt}{2ex}
	\adjustbox{valign=t}{
		\begin{tabular}{ c }
		\includegraphics[width=0.6\textwidth, margin=-10pt 1ex -10pt 1ex, valign=m]{solsys_vesta.jpg}\\
		\end{tabular}
		}
	& \vfill
	Vesta is the second-largest asteroid belt object after Ceres. It was visited by the NASA Dawn spacecraft in 2011.\newline
	\newline
	\textit{Surface and elevation data for Orbiter are derived from Dawn HAMO mosaics.}\\
	\hline
	\end{tabularx}
\end{table}


\begin{table}[H]
	\begin{tabularx}{\textwidth}{ |lX| }
	\hline\rule{0pt}{2ex}
	\textbf{Jupiter} &\\
	\hline\rule{0pt}{2ex}
	\adjustbox{valign=t}{
		\begin{tabular}{ c }
		\includegraphics[width=0.6\textwidth, margin=-10pt 1ex -10pt 1ex, valign=m]{solsys_jupiter.jpg}\\
			\adjustbox{valign=t}{
			\begin{tabular}{ ll }
			\includegraphics[width=0.285\textwidth, margin=-20pt 1ex 0pt 1ex, valign=m]{solsys_io.jpg} &
			\includegraphics[width=0.285\textwidth, margin=0pt 1ex -20pt 1ex, valign=m]{solsys_europa.jpg}\\
			\includegraphics[width=0.285\textwidth, margin=-20pt 1ex 0pt 1ex, valign=m]{solsys_ganymede.jpg} &
			\includegraphics[width=0.285\textwidth, margin=0pt 1ex -20pt 1ex, valign=m]{solsys_callisto.jpg}\\
			\end{tabular}
			}
		\end{tabular}
		}
	& \vfill
	Jupiter is the 5$^{th}$ planet from the Sun, and the innermost of the gas giants. It is the most massive of the planets in the solar system. Its mass allowed it to retain the lighter elements; it is mainly composed of hydrogen and helium, but probably has a core of heavier elements. Jupiter's atmosphere has a characteristic band structure. Its most prominent feature is the Great Red Spot, an anticyclonic storm that has persisted for centuries.\newline
	Jupiter has a faint ring system and is orbited by at least 95 moons, four of which (the Galilean moons, so called because they were discovered by Galileo Galilei in 1610) are modelled in Orbiter.\newline
	Jupiter and its moons have been visited by several robotic missions, starting with Pioneer 10 and including the Voyager probes, Galileo, Cassini, New Horizons and Juno.\newline
	\newline
	\textit{In Orbiter, Jupiter visuals are based on Cassini image data (NASA/JPL/Space Science Institute), with cloud maps by Rolf Keibel based on CICLOPS maps.\newline
%TODO move ref to references?
	The Galilean moon maps are based on Galileo/SSI/Voyager mosaics (Astrogeology Science Center, USGS), with Io elevations based on O. L. White et al., J. Geophys. Res.: Planets 119(6), 1276-1301 (2014)}\\
	\hline
	\end{tabularx}
\end{table}


\begin{table}[H]
	\begin{tabularx}{\textwidth}{ |lX| }
	\hline\rule{0pt}{2ex}
	\textbf{Saturn} &\\
	\hline\rule{0pt}{2ex}
	\adjustbox{valign=t}{
		\begin{tabular}{ c }
		\includegraphics[width=0.6\textwidth, margin=-10pt 1ex -10pt 1ex, valign=m]{solsys_saturn.jpg}\\
			\adjustbox{valign=t}{
			\begin{tabular}{ lll }
			\includegraphics[width=0.18\textwidth, margin=-20pt 1ex 0pt 1ex, valign=m]{solsys_mimas.jpg} &
			\includegraphics[width=0.18\textwidth, margin=0pt 1ex 0pt 1ex, valign=m]{solsys_enceladus.jpg} &
			\includegraphics[width=0.18\textwidth, margin=0pt 1ex -20pt 1ex, valign=m]{solsys_tethys.jpg} \\
			\includegraphics[width=0.18\textwidth, margin=-20pt 1ex 0pt 1ex, valign=m]{solsys_dione.jpg} &
			\includegraphics[width=0.18\textwidth, margin=0pt 1ex 0pt 1ex, valign=m]{solsys_rhea.jpg} &
			\includegraphics[width=0.18\textwidth, margin=0pt 1ex -20pt 1ex, valign=m]{solsys_titan.jpg}\\
			\includegraphics[width=0.18\textwidth, margin=-20pt 1ex 0pt 1ex, valign=m]{solsys_hyperion.jpg} &
			\includegraphics[width=0.18\textwidth, margin=0pt 1ex -20pt 1ex, valign=m]{solsys_iapetus.jpg} &\\
			\end{tabular}
			}
		\end{tabular}
		}
	& \vfill
	Saturn is one of the gas giants and the second-largest planet in the solar system. Like Jupiter, it is composed predominantly of hydrogen and helium. Saturn's most striking feature is the ring system in the equatorial plane, composed mostly of water ice particles and stretching out from about 1.1 to 3 planet radii, with a thickness of about 20 km.\newline
	Saturn has 146 known moons, not counting a large number of moonlets in the rings. Titan is of comparable size to Mercury and the only moon in the solar system with a substantial atmosphere. There is complex gravitational interaction between the moons and rings, with shepherd moons clearing gaps and keeping rings contained, as well as causing ripples in the ring structure.\newline
	\newline
%TODO move ref to references?
	\textit{Saturn's surface and rings in Orbiter are based on maps by Björn Jónsson, adapted by Rolf Keibel. The moon surfaces based on NASA/JPL-Caltech/SSI/Lunar and Planetary Institute, Cassini Central Laboratory for Operations. Titan elevation data P. Corlies et al, Geophysical Research Letters 44(23), 11754-11761 (2017)}\\
	\hline
	\end{tabularx}
\end{table}


\begin{table}[H]
	\begin{tabularx}{\textwidth}{ |lX| }
	\hline\rule{0pt}{2ex}
	\textbf{Uranus} &\\
	\hline\rule{0pt}{2ex}
	\adjustbox{valign=t}{
		\begin{tabular}{ c }
		\includegraphics[width=0.6\textwidth, margin=-10pt 1ex -10pt 1ex, valign=m]{solsys_uranus.jpg}\\
			\adjustbox{valign=t}{
			\begin{tabular}{ ll }
			\includegraphics[width=0.285\textwidth, margin=-20pt 1ex 0pt 1ex, valign=m]{solsys_miranda.jpg} &
			\includegraphics[width=0.285\textwidth, margin=0pt 1ex -20pt 1ex, valign=m]{solsys_ariel.jpg}\\
			\includegraphics[width=0.285\textwidth, margin=-20pt 1ex 0pt 1ex, valign=m]{solsys_umbriel.jpg} &
			\includegraphics[width=0.285\textwidth, margin=0pt 1ex -20pt 1ex, valign=m]{solsys_titania.jpg}\\
			\includegraphics[width=0.285\textwidth, margin=-20pt 1ex 0pt 1ex, valign=m]{solsys_oberon.jpg} &\\
			\end{tabular}
			}
		\end{tabular}
		}
	& \vfill
	Uranus is the seventh planet from the Sun, third-largest by radius and fourth by mass. Its chemical composition is similar to Neptune, containing a solid core of silicate and iron-nickel, a mantle of water/ammonia/methane ice and a hydrogen/helium atmosphere.\newline
	The planet produces little internal heat and emits virtually no excess energy, making it the coldest planet in the solar system.\newline
	Uranus has a ring system consisting of very dark particles and is orbited by at least 28 moons, the main 5 of which are modelled in Orbiter.\newline
	\newline
	\textit{Uranus textures are based on maps by James Hastings-Trew, adapted for Orbiter by Rolf Keibel. The moon textures are based on maps by Robert Stettner and Voyager images adapted by Rolf Keibel.}\\
	\hline
	\end{tabularx}
\end{table}


\begin{table}[H]
	\begin{tabularx}{\textwidth}{ |lX| }
	\hline\rule{0pt}{2ex}
	\textbf{Neptune} &\\
	\hline\rule{0pt}{2ex}
	\adjustbox{valign=t}{
		\begin{tabular}{ c }
		\includegraphics[width=0.6\textwidth, margin=-10pt 1ex -10pt 1ex, valign=m]{solsys_neptune.jpg}\\
			\adjustbox{valign=t}{
			\begin{tabular}{ ll }
			\includegraphics[width=0.285\textwidth, margin=-20pt 1ex 0pt 1ex, valign=m]{solsys_triton.jpg} &
			\includegraphics[width=0.285\textwidth, margin=0pt 1ex -20pt 1ex, valign=m]{solsys_proteus.jpg}\\
			\includegraphics[width=0.285\textwidth, margin=-20pt 1ex 0pt 1ex, valign=m]{solsys_nereid.jpg} &\\
			\end{tabular}
			}
		\end{tabular}
		}
	& \vfill
	Neptune is the farthest planet from the Sun. It is one of the ice giants, and similar in size and composition to Uranus, with a solid rocky core, a mantle of water, ammonia and methane ice, and an atmosphere of hydrogen, helium and methane. The atmosphere frequently features storm systems, visible as dark spots, which can persist for several months and are often accompanied by bright clouds higher up in the atmosphere.\newline
	Neptune has 16 known moons. It also exerts a gravitational influence on the Kuiper belt objects in the region beyond its orbit, including Pluto which orbits in a 2:3 resonance with Neptune.\newline
	\newline
	\textit{Orbiter's Neptune textures based on a map by James Hastings-Trew, moons by Robert Stettner, from Planetary Mean Orbital Parameters and Moon Maps.}\\
	\hline
	\end{tabularx}
\end{table}

\subsection{Selected constants and parameters}
These data may be useful when planning missions across the solar system.

\subsubsection{Astrodynamic constants}

%\begin{table}[H]
	%\centering
	\begin{longtable}{ |p{0.35\textwidth}|p{0.14\textwidth}|p{0.42\textwidth}| }
	\hline\rule{0pt}{2ex}
	\textbf{Constant} & \textbf{Symbol} & \textbf{Value}\\
	\hline\rule{0pt}{2ex}
	Julian day & d & 86400 s\\
	\hline\rule{0pt}{2ex}
	Julian year & yr & 365.25 d\\
	\hline\rule{0pt}{2ex}
	Julian century & Cy & 36525 d\\
	\hline\rule{0pt}{2ex}
	Speed of light & c & 299792458 m/s\\
	\hline\rule{0pt}{2ex}
	Gaussian gravitational constant & k & 0.01720209895 (AU$^{3}$/d$^{2}$)$^{1/2}$\\
	\hline
	\caption{Defining constants}
	\end{longtable}
%\end{table}

%TODO add vesta, ceres, etc
%\begin{table}[H]
	%\centering
	\begin{longtable}{ |p{0.35\textwidth}|p{0.14\textwidth}|p{0.42\textwidth}| }
	\hline\rule{0pt}{2ex}
	\textbf{Constant} & \textbf{Symbol} & \textbf{Value}\\
	\hline\rule{0pt}{2ex}
	Mean sidereal day & & 86164.09054 s = 23:56:04.09054\\
	\hline\rule{0pt}{2ex}
	Sidereal year (quasar ref. frame) & & 365.25636 d\\
	\hline\rule{0pt}{2ex}
	Light time for 1 AU & $\tau_{A}$ & 499.004783806 ($\pm$ 0.00000001) s\\
	\hline\rule{0pt}{2ex}
	Gravitational constant & G & 6.67259 ($\pm$ 0.00030) $\times$ 10$^{-11}$ kg$^{-1}$ m$^{3}$ s$^{-2}$\\
	\hline\rule{0pt}{2ex}
	General precession in longitude & & 5028.83 ($\pm$ 0.04) arcsec/Cy\\
	\hline\rule{0pt}{2ex}
	Obliquity of ecliptic (J2000.0) & $\epsilon$ & 84381.412 ($\pm$ 0.005) arcsec\\
	\hline\rule{0pt}{2ex}
	Mass: Sun / Mercury & & 6023600. ($\pm$ 250.)\\
	\hline\rule{0pt}{2ex}
	Mass: Sun / Venus & & 408523.71 ($\pm$ 0.06)\\
	\hline\rule{0pt}{2ex}
	Mass: Sun / (Earth + Moon) & & 328900.56 ($\pm$ 0.02)\\
	\hline\rule{0pt}{2ex}
	Mass: Sun / (Mars system) & & 3098708. ($\pm$ 9.)\\
	\hline\rule{0pt}{2ex}
	Mass: Sun / (Jupiter system) & & 1047.3486 ($\pm$ 0.0008)\\
	\hline\rule{0pt}{2ex}
	Mass: Sun / (Saturn system) & & 3497.898 ($\pm$ 0.018)\\
	\hline\rule{0pt}{2ex}
	Mass: Sun / (Uranus system) & & 22902.98 ($\pm$ 0.03)\\
	\hline\rule{0pt}{2ex}
	Mass: Sun / (Neptune system) & & 19412.24 ($\pm$ 0.04)\\
	\hline\rule{0pt}{2ex}
	Mass: Sun / (Pluto system) & & 1.35 ($\pm$ 0.07) $\times$ 10$^{8}$\\
	\hline\rule{0pt}{2ex}
	Mass: Moon / Earth & & 0.012300034 ($\pm$ 3 $\times$ 10$^{-9}$)\\
	\hline
	\caption{Primary constants}
	\end{longtable}
%\end{table}

%\begin{table}[H]
	%\centering
	\begin{longtable}{ |p{0.35\textwidth}|p{0.14\textwidth}|p{0.42\textwidth}| }
	\hline\rule{0pt}{2ex}
	\textbf{Constant} & \textbf{Symbol} & \textbf{Value}\\
	\hline\rule{0pt}{2ex}
	Astronomical unit distance & c $\times$ $\tau_{A}$ = AU & 1.49597870691 $\times$ 10$^{11}$ ($\pm$ 3) m\\
	\hline\rule{0pt}{2ex}
	Heliocentric gravitational constant & k$^{2}$ AU$^{3}$ d$^{-2}$ = GM$_{SUN}$ & 1.32712440018 $\times$ 10$^{20}$ ($\pm$ 8 $\times$ 10$^{9}$) m$^{3}$ s$^{-2}$\\
	\hline\rule{0pt}{2ex}
	Mass: Earth / Moon & & 81.30059 ($\pm$ 0.00001)\\
	\hline
	\caption{Derived constants}
	\end{longtable}
%\end{table}

\noindent
All values from \cite{standish1995}\\
\\
\underline{Notes:}\\
Data are from the 1994 IAU file of current best estimates. Planetary ranging determines the Earth/Moon mass ratio. The value for 1 AU is taken from JPL's current planetary ephemeris DE-405.\\
\\
\textbf{Planetary mean orbital elements and centennial rates}

%TODO add vesta, ceres, etc
%\begin{table}[H]
	%\centering
	\begin{longtable}{ |p{0.1\textwidth}|p{0.13\textwidth}|p{0.13\textwidth}|p{0.1\textwidth}|p{0.12\textwidth}|p{0.11\textwidth}|p{0.11\textwidth}| }
	\hline\rule{0pt}{2ex}
	\textbf{Planet} & \textbf{a [AU]} & \textbf{e} & \textbf{i [deg]} & \textbf{$\Omega$ [deg]} & \textbf{$\bar{\omega}$ [deg]} & \textbf{L [deg]}\\
	\hline\rule{0pt}{2ex}
	Mercury & 0.38709893 & 0.20563069 & 7.00487 & 48.33167 & 77.45645 & 252.25084\\
	\hline\rule{0pt}{2ex}
	Venus & 0.72333199 & 0.00677323 & 3.39471 & 76.68069 & 131.53298 & 181.97973\\
	\hline\rule{0pt}{2ex}
	Earth & 1.00000011 & 0.01671022 & 0.00005 & -11.26064 & 102.94719 & 100.46435\\
	\hline\rule{0pt}{2ex}
	Mars & 1.52366231 & 0.09341233 & 1.85061 & 49.57854 & 336.04084 & 355.45332\\
	\hline\rule{0pt}{2ex}
	Jupiter & 5.20336301 & 0.04839266 & 1.30530 & 100.55615 & 14.75385 & 34.40438\\
	\hline\rule{0pt}{2ex}
	Saturn & 9.53707032 & 0.05415060 & 2.48446 & 113.71504 & 92.43194 & 49.94432\\
	\hline\rule{0pt}{2ex}
	Uranus & 19.19126393 & 0.04716771 & 0.76986 & 74.22988 & 170.96424 & 313.23218\\
	\hline\rule{0pt}{2ex}
	Neptune & 30.06896348 & 0.00858587 & 1.76917 & 131.72169 & 44.97135 & 304.88003\\
	\hline\rule{0pt}{2ex}
	Pluto & 39.48168677 & 0.24880766 & 17.14175 & 110.30347 & 224.06676 & 238.92881\\
	\hline
	\end{longtable}
%\end{table}

%TODO add vesta, ceres, etc
%\begin{table}[H]
	%\centering
	\begin{longtable}{ |p{0.1\textwidth}|p{0.13\textwidth}|p{0.13\textwidth}|p{0.08\textwidth}|p{0.11\textwidth}|p{0.1\textwidth}|p{0.15\textwidth}| }
	\hline\rule{0pt}{2ex}
	\textbf{Planet} & \textbf{da/dt [AU/Cy]} & \textbf{de/dt [/Cy]} & \textbf{di/dt [''/Cy]} & \textbf{d$\Omega$/dt [''/Cy]} & \textbf{d$\bar{\omega}$/dt [''/Cy]} & \textbf{dL/dt [''/Cy]}\\
	\hline\rule{0pt}{2ex}
	Mercury & 0.00000066 & 0.00002527 & -23.51 & -446.30 & 573.57 & 538101628.29\\
	\hline\rule{0pt}{2ex}
	Venus & 0.00000092 & -0.00004938 & -2.86 & -996.89 & -108.80 & 210664136.06\\
	\hline\rule{0pt}{2ex}
	Earth & -0.00000005 & -0.00003804 & -46.94 & -18228.25 & 1198.28 & 129597740.63\\
	\hline\rule{0pt}{2ex}
	Mars & -0.00007221 & 0.00011902 & -25.47 & -1020.19 & 1560.78 & 68905103.78\\
	\hline\rule{0pt}{2ex}
	Jupiter & 0.00060737 & -0.00012880 & -4.15 & 1217.17 & 839.93 & 10925078.35 \\
	\hline\rule{0pt}{2ex}
	Saturn & -0.00301530 & -0.00036762 & 6.11 & -1591.05 & -1948.89 & 4401052.95\\
	\hline\rule{0pt}{2ex}
	Uranus & 0.00152025 & -0.00019150 & -2.09 & -1681.40 & 1312.56 & 1542547.79\\
	\hline\rule{0pt}{2ex}
	Neptune & -0.00125196 & 0.0000251 & -3.64 & -151.25 & -844.43 & 786449.21\\
	\hline\rule{0pt}{2ex}
	Pluto & -0.00076912 & 0.00006465 & 11.07 & -37.33 & -132.25 & 522747.90\\
	\hline
	\end{longtable}
%\end{table}

\noindent
a: semi-major axis, e: eccentricity, i: inclination, $\Omega$: longitude of the ascending node, $\bar{\omega}$: longitude of perihelion, L: mean longitude, Cy: Julian century, '': arcsec\\
Epoch = J2000 = 2000 January 1 12h\\
All values from \cite{seidelmann1992p316}\\
\\
\underline{Notes:}\\
The tables contain mean orbit solutions from a 250 yr. least squares fit of the DE 200 planetary ephemeris to a Keplerian orbit where each element is allowed to vary linearly with time. This solution fits the terrestrial planet orbits to $\sim$25'' or better, but achieves only $\sim$600'' for Saturn. Elements are referenced to mean ecliptic and equinox of J2000 at the J2000 epoch (2451545.0 JD).

\subsubsection{Selected physical parameters}

%TODO add vesta, ceres, etc
%\begin{table}[H]
	%\centering
	\begin{longtable}{ |p{0.1\textwidth}|p{0.16\textwidth}|p{0.15\textwidth}|p{0.17\textwidth}|p{0.13\textwidth}|p{0.12\textwidth}| }
	\hline\rule{0pt}{2ex}
	\textbf{Planet} & \textbf{mean radius [km]} & \textbf{mass [10$^{23}$ km]} & \textbf{density [g/cm$^{3}$]} & \textbf{sidereal rotation period [h]} & \textbf{sidereal orbit period [yr]}\\
	\hline\rule{0pt}{2ex}
	Mercury & 2440. $\pm$ 1. & 3.301880 & 5.427 & 1407.509 & 0.2408445\\
	\hline\rule{0pt}{2ex}
	Venus & 6051.84 $\pm$ 0.01 & 48.6855374 & 5.204 & -5832.444 & 0.6151826\\
	\hline\rule{0pt}{2ex}
	Earth & 6371.01 $\pm$ 0.02 & 59.73698968 & 5.515 & 23.93419** & 0.9999786\\
	\hline\rule{0pt}{2ex}
	Mars & 3389.92 $\pm$ 0.04 & 6.418542 & 3.9335 $\pm$ 0.0004 & 24.622962 & 1.88071105\\
	\hline\rule{0pt}{2ex}
	Jupiter & 69911. $\pm$ 6. & 18986.111 & 1.326 & 9.92425 & 11.856523\\
	\hline\rule{0pt}{2ex}
	Saturn & 58232. $\pm$ 6. & 5684.6272 & 0.6873 & 10.65622 & 29.423519\\
	\hline\rule{0pt}{2ex}
	Uranus & 25362. $\pm$ 12. & 868.32054 & 1.318 & 17.24 $\pm$ 0.01 & 83.747407\\
	\hline\rule{0pt}{2ex}
	Neptune & 24624. $\pm$ 21. & 1024.569 & 1.638 & 16.11 $\pm$ 0.01 & 163.72321\\
	\hline\rule{0pt}{2ex}
	Pluto* & 1151 & 0.15 & 1.1 & 153.28 & 248.0208\\
	\hline
	\end{longtable}
%\end{table}

%TODO add vesta, ceres, etc
%\begin{table}[H]
	%\centering
	\begin{longtable}{ |p{0.1\textwidth}|p{0.17\textwidth}|p{0.17\textwidth}|p{0.21\textwidth}|p{0.21\textwidth}| }
	\hline\rule{0pt}{2ex}
	\textbf{Planet} & \textbf{V(1,0) [mag.]} & \textbf{Geometric albedo} & \textbf{Equatorial gravity [m/s$^{2}$]} & \textbf{Escape velocity [km/s]}\\
	\hline\rule{0pt}{2ex}
	Mercury & -0.42 & 0.106 & 3.701 & 4.435\\
	\hline\rule{0pt}{2ex}
	Venus & -4.4 & 0.65 & 8.87 & 10.361\\
	\hline\rule{0pt}{2ex}
	Earth & -3.86 & 0.367 & 9.780327 & 11.186\\
	\hline\rule{0pt}{2ex}
	Mars & -1.52 & 0.15 & 3.69 & 5.027\\
	\hline\rule{0pt}{2ex}
	Jupiter & -9.4 & 0.52 & 23.12 $\pm$ 0.01 & 59.5\\
	\hline\rule{0pt}{2ex}
	Saturn & -8.88 & 0.47 & 8.96 $\pm$ 0.01 & 35.5\\
	\hline\rule{0pt}{2ex}
	Uranus & -7.19 & 0.51 & 8.69 $\pm$ 0.01 & 21.3\\
	\hline\rule{0pt}{2ex}
	Neptune & -6.87 & 0.41 & 11.00 $\pm$ 0.05 & 23.5\\ 
	\hline\rule{0pt}{2ex}
	Pluto* & -1.0* & 0.3* & 0.655 & 1.3\\
	\hline
	\end{longtable}
%\end{table}

\noindent
All values from \cite{yoder1995} except Pluto data from \cite{seidelmann1992p706}. Mercury to Neptune masses derived from standard gravitational parameter (GM) data in \cite{yoder1995}.\\
** Orbiter uses 23.93447 h (=23 h 56 m 4.09 s) which gives better long-term stability.

\end{document}
